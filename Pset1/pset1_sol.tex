\documentclass[12pt]{article}
\usepackage{amsmath,amsfonts,bbm,xfrac}
\usepackage{fancyhdr,enumitem,xcolor,subcaption}
\usepackage{graphicx} % Allows including images
\usepackage[left=3cm, right=3cm, top =2cm, bottom = 3cm]{geometry}
\newcommand{\E}{\mathbb{E}}


\pagestyle{plain}
%\setcounter{secnumdepth}{0}
\pagestyle{fancy} 
\rhead{Winter 2023} 
\chead{} 
\lhead{Econ 33550 - Spatial Economics} 
\lfoot{} 
\cfoot{\thepage} 
\rfoot{} 


\title{Spatial Economics - Problem Set I}
\author{Jose M. Quintero \and Jun Wong \and Rachel Williams}

\begin{document}

\maketitle

\section{Introduction}

Some blah blah of what are we going to do, summarize the ingredients of the model, mention the data, limitations of the data if any, our counterfactual and main findings. This should be relatively fast. 

\section{Analysis of Chicago}

A lot of blah, but most critical excellent maps should go here! Beautiful and colorful. 

\section{The Model}


\subsection{Households}
A household is characterized by the neighborhood where she resides $i$ and where she works $j$. The household consume final good $c_{ij}$, land $H_{ij}$. Their utility is enhanced by neighborhood-specific amenities $B_i(R_i)$. Agents additionally have an idiosyncratic shock to their preferences $z_{ij}$ and $\kappa_{ij}\geq 1$ is the disutility of commuting from $i$ to $j$.  \\ 
\begin{equation}
    U_{ij}(z_{ij}) = \frac{B_iz_{ij}}{\kappa_{ij}}\left(\frac{c_{ij}}{\beta}\right)^{\beta}\left(\frac{H_{ij}}{1-\beta}\right)^{1-\beta}
\end{equation}
Following the trade literature, we assume that the shock is distributed Fréchet 
\begin{align*}
    \textcolor{blue}{F(s_{ij})} &\textcolor{blue}{= e^{\lambda_{ij}s_{ij}^{-\theta}}} & \textcolor{red}{F(s_{ij})} &\textcolor{red}{= e^{\lambda^o_{i}\lambda^d_js_{ij}^{-\theta}}}
\end{align*}
with $\theta>0$. 
The budget restriction for the agent is 
\begin{equation*}
    c + q_i^{r}H_{ij} = w_j
\end{equation*}
where I am normalizing the price of the consumption good and assume it is freely tradable within the city. Using profit maximization since the utility is a Cobb-Douglas, then the optimal decisions for $c$ and $H$ are  
\begin{align*}
    c_{ij} &= \beta w_j & H_{ij}&= \boxed{ \frac{(1-\beta)w_j}{q_i^{r}}}
\end{align*}
Then the utility for an agent living in neighborhood $i$, working in $j$ with a preference shock is 
\begin{equation}
    U_{ij}(z_{ij}) = \frac{B_iz_{ij}w_j}{\kappa_{ij}(q_{i}^r)^{1-\beta}}
\end{equation}
The intuition of the equation is pretty standard. Higher wages yield higher utility, higher commuting cost decreases utility, amenities increase utility, prices decrease utility and the preference shock increases utility. 

\subsection{Commuting Flows}
First, notice that the utility is also distributed Fréchet, 
\begin{align*}
    \Pr\left(U_{ij}\leq u\right) &= \Pr\left(\frac{B_iz_{ij}w_j}{\kappa_{ij}q_{i}^{1-\beta}}\leq u\right) 
    = \Pr\left( z_{ij}\leq \frac{u\kappa_{ij}q_i^{1-\beta}}{B_iw_j}\right) \\ 
    &= \exp\left(-\lambda_{ij}\left(\frac{B_iz_{ij}w_j}{\kappa_{ij}q_{i}^{1-\beta}}\right)^\theta u^{-\theta}\right) \\ 
    &= \exp\left(-\Phi_{ij}u^{-\theta}\right) = G_{ij}(u)
\end{align*}
Given this fact, the probability that an agent chooses to live in location $i$ and work in location $j$ is 
\begin{align*}
    \Pr\left(U_{ij}\geq \max_{r,s} U_{rs}\right) &= \Pr\left(U_{ij}\geq \max_{s} U_{is}\right)\Pr\left(U_{ij}\geq \max_{r} U_{kr},\quad\forall k\right) \\ 
    &= \prod_{s\neq i}\Pr\left( U_{ij}\geq U_{is}\right)\left[\prod_{r\neq j}\prod_{k}\Pr\left(U_{ij}\geq U_{kr}\right)\right] \\
    &= \int_0^{\infty}\theta\Phi_{ij}u^{-(\theta+1)}\prod_{s}\prod_{r}e^{-\Phi_{rs}u^{-\theta}} \mathrm{d}u \\ 
    &= \int_0^{\infty}\theta\Phi_{ij}u^{-(\theta+1)}e^{-\sum_{r}\sum_{s}\Phi_{rs}u^{-\theta}} \mathrm{d}u \\ 
    &= \frac{\Phi_{ij}}{\Phi}\int_0^{\infty}\theta\Phi u^{-(\theta+1)}e^{-\Phi u^{-\theta}} \mathrm{d}u \\ 
    &= \frac{\Phi_{ij}}{\Phi} = \pi_{ij}
\end{align*}
where $\Phi=\sum_{r}\sum_{s}\Phi_{rs}$. Using this result the probability an agent resides in location $i$ is 
\begin{equation*}
    \pi_{Ri} = \sum_{j=1}^N \pi_{ij} = \frac{1}{\Phi}\sum_{j=1}^N \Phi_{ij}
\end{equation*}
Similarly, the probability an agent works in location $j$ is
\begin{equation*}
    \pi_{Wj} = \sum_{i=1}^N \pi_{ij} = \frac{1}{\Phi}\sum_{i=1}^N \Phi_{ij}
\end{equation*}
Similar, conditional to living in location $i$, the probability of commuting to location $j$ is
\begin{align*}
    \pi_{ij\vert i} = \Pr\left(U_{ij}\geq \max_{r}U_{ir}\right) &= \Pr\left(\frac{B_iw_jz_{ij}}{d_{ij}q_i^{1-\beta}} \geq \max_{r} \frac{B_iw_rz_{ir}}{d_{ir}q_i^{1-\beta}} \right) \\ 
    &= \Pr\left(\frac{w_jz_{ij}}{d_{ij}} \geq \max_{r} \frac{w_rz_{ir}}{d_{ir}} \right) \\ 
    &= \int_0^\infty\Pr\left(\frac{w_jz_{ij}}{d_{ij}} \geq \max_{r} \frac{w_rz_{ir}}{d_{ir}}\bigg\vert z_{ij} \right) \mathrm{d}G_{ij}(z_{ij}) \\ 
    &= \int_0^\infty\prod_{r\neq j}\Pr\left(z_{ir}\leq \frac{w_jd_{ir}}{w_rd_{ij}}\bigg\vert z_{ij} \right) \mathrm{d}G_{ij}(z_{ij}) \\ 
    % &= \int_0^\infty \prod_{r=1}^N\exp\left(-\lambda_{ir}\left(\frac{w_jd_{ir}}{w_rd_{ij}}\right)^{-\theta}z_{ij}^{-\theta}\right)\lambda_{ij}\theta z_{ij}^{-(\theta+1)}\mathrm{d}z_{ij} \\ 
    &= \int_0^\infty \exp\left(-\left(\frac{w_j}{d_{ij}}\right)^{-\theta}z_{ij}^{-\theta}\sum_{r=1}^N\lambda_{ir}\left(\frac{d_{ir}}{w_r}\right)^{-\theta}z_{ij}^{-\theta}\right)\lambda_{ij}\theta z_{ij}^{-(\theta+1)}\mathrm{d}z_{ij} \\ 
    &= \frac{\lambda_{ij}\left(\sfrac{w_j}{d_{ij}}\right)^\theta}{\sum_{r=1}^N \lambda_{ir}\left(\sfrac{w_r}{d_{ir}}\right)^{\theta}}
\end{align*}
This expression allows to calculate the average wage of people residing in location $i$
\begin{equation*}
    \E\left[w_{Ri} \right] = \boxed{\sum_{j=1}^n w_j \pi_{ij\vert i}}
\end{equation*}

\subsection{Firms}

Firms produce a tradable good using a combination of labor $L_j$ and floor space $H_j^p$. To keep production simple, assume that the firms combine both inputs using a Cobb-Douglas production function with constant returns to scale
\begin{equation}
    Y_j = A_jL_j^\alpha \left(H_j^p\right)^{1-\alpha}, \quad\alpha\in(0,1)
\end{equation}
Let $y_t$ be the normalized output per unit of land, and $w_j$ the wage paid in location $j$. Then, the optimization problem for the firm is 
\begin{equation*}
    \max_{\ell_j} A_j\ell_j^\alpha - w_j\ell_j
\end{equation*}
The FOC implies the following wage
\begin{equation}
    w_j = \alpha A_j\ell_j^{\alpha-1}
\end{equation}
Assuming that $1-\alpha>\gamma$, the labor demand function for location $j$ is decreasing and has the form: 
\begin{equation*}
    L_j = \left(\frac{A_j\alpha}{w_j}\right)^{\frac{1}{1-\alpha}}H_j^p
\end{equation*}
Finally, as the firm has constant returns to scale, it will rent land until it hits zero profits. Then the demand for floor space it 
\begin{align*}
    q_j^p &= \frac{Y_j - w_jL_j}{H^p_j} \\ 
    &= (1-\alpha)A_j\ell_j^{\alpha} \\ 
    &= \boxed{(1-\alpha)A_j\left(\frac{\alpha A_j}{w_j}\right)^{\frac{\alpha}{1-\alpha}}}
\end{align*}

\subsection{Floor Space}
For this section, we borrow the structure from the Berlin paper. Let $\theta_j$ be the share of floor space use for production. Then, for there not to be arbitrage, it has to be true that 
\begin{equation*}
    \theta_i = \begin{cases}
    1 &\mbox{if } q_i^p>q_i^r \\ 
    \in[0,1] &\mbox{if } q_i^p=q_i^r \\
    0 &\mbox{if } q_i^p<q_i^r 
    \end{cases}
\end{equation*}
Let $Q_i=\max\{q_i^p,q_i^r\}$ be the observed price for floor space. Following the standard assumptions of urban models, floor space is supplied by competitive landlords. Specifically, floor space is produced using capital and land using a Cobb-Douglas aggregator
\begin{equation*}
    H_i = M_i^\mu K_i^{1-\mu}
\end{equation*}
Denote by $P$ the price for capital which is common across the city and $R_i$ the price for land. FOC imply that 
\begin{align*}
    \frac{\mu}{1-\mu}\frac{K_i}{M_i} = \frac{P}{R_i}
\end{align*}
By substituting $\sfrac{K_i}{M_i}$ in any of the FOC from the landlord it follows that
\begin{equation*}
    \boxed{Q_i = \underbrace{\mu^{-\mu}(1-\mu)^{1-\mu}P^{-\mu}}_{\chi}R_i^{1-\mu} }
\end{equation*}
Similarly, the total supply of floor space can be written as 
\begin{equation*}
    \boxed{H_i = \underbrace{\varphi_i}_{M_i^\mu}K_i^{1-\mu}}
\end{equation*}
Finally, the total demand of labor by the firm in location should equate the commuting flows
\begin{equation*}
    L_j = \sum_{i=1}^N \pi_{ij\vert i}L_{Ri}
\end{equation*}

\subsection{Market clearing conditions}
First, residential has to clear. Using the optimal demand from the household, \textcolor{red}{I want to write this as the definition of the equilibrium. This is lacking a little structure now but I just wanted to make sure all the pieces are here before moving to the extension. }
\begin{equation}
    (1-\theta_i)H_{Ri} = \frac{(1-\beta)L_{Ri}\E\left[w_{Ri}\right]}{q_i^r}
\end{equation}
where $L_{Ri}$ is the number of residents in location $i$. Similarly, the market for floor space use for production has to clear and satisfies
\begin{equation}
    \theta_i H_i^p = \left(\frac{(1-\alpha)A_i}{q_i}\right)^{\sfrac{1}{\alpha}}L_i
\end{equation}
Additionally, the total supply of floor space has to clear. That is 
\begin{equation}
    \varphi_iK_i^{1-\mu} = \theta H_i^p + (1-\theta)H_{Ri}
\end{equation}
Finally

\subsection{Extension: Two types}
We now consider a version where there are two types of agents $m\in\{s,u\}$. Let $\delta=\Pr\left(m=s\right)$. Also, assume that the type of the agent is independent from the preference shock. We assume both types of agents are required for production and they are imperfect substitutes. Specifically, we now assume the firms production function is 
\begin{equation*}
    Y_j = A_j L_j^\alpha H_j^{1-\alpha}\quad\text{s.t}\quad L_j = L_{js}^\psi L_{ju}^{1-\psi}
\end{equation*}
Then the full problem of the firm is 
\begin{equation*}
    \max_{L_{sj},L_{uj},H_j}  A_j L_j^\alpha H_j^{1-\alpha}-w_sL_{sj}-w_uL_{uj}-q_j^pH_j^p\quad\text{s.t}\quad L_j = L_{js}^\psi L_{ju}^{1-\psi}
\end{equation*}
Consider again the normalized version of the problem. For the time sake I drop the location subindex and later bring it back again. The FOC are 
\begin{align*}
    [\ell_s]&: & w_s&= A\alpha\left[\left(\frac{\ell_s}{\ell_u}\right)^{\psi}\ell_u\right]^{\alpha-1}\psi \left(\frac{\ell_u}{\ell_s}\right)^{1-\psi}  & \\ 
    [\ell_u]&: & w_u&= A\alpha\left[\left(\frac{\ell_s}{\ell_u}\right)^{\psi}\ell_u\right]^{\alpha-1}(1-\psi) \left(\frac{\ell_u}{\ell_s}\right)^{-\psi} &
\end{align*}
Combining both FOC it follows that
\begin{equation*}
    \frac{\ell_u}{\ell_s} = \underbrace{\frac{w_s}{w_u}\frac{1-\psi}{\psi}}_{\Gamma}
\end{equation*}
The previous equation gives some interesting intuition as the wage premium depends on the price elasticity of each type of labor, governed by $\psi$ and the relative demand. Moreover, the relative demand is constrain within a location which allows to solve for the quantities of labor for each type. By plugging back into the FOC it follows that 
\begin{equation*}
    \ell_u = \left(A\alpha\right)^{\frac{1}{1-\alpha}}\left(\frac{\psi}{w_s}\right)^{\frac{\alpha\psi}{1-\alpha}} \left(\frac{1-\psi}{w_u}\right)^{\frac{1-\psi\alpha}{1-\alpha}}
\end{equation*}
Similarly, 
\begin{equation*}
    \ell_s = \left(A\alpha\right)^{\frac{1}{1-\alpha}}\left(\frac{\psi}{w_s}\right)^{\frac{1-\alpha(1-\psi)}{1-\alpha}} \left(\frac{1-\psi}{w_u}\right)^{\frac{\alpha(1-\psi)}{1-\alpha}}
\end{equation*}
and 
\begin{equation*}
    \ell = (A\alpha)^{\frac{1}{1-\alpha}}\left(\frac{\psi}{w_s}\right)^{\frac{\psi}{1-\alpha}}\left(\frac{1-\psi}{w_u}\right)^{\frac{1-\psi}{1-\alpha}}
\end{equation*}
Note that first-order conditions imply that 
\begin{equation*}
    w_s\ell_s +  w_u\ell_u = A\alpha\ell^\alpha
\end{equation*}
Then using the zero profit conditions, the price for land is 
\begin{equation*}
    q_i^p = A_i(1-\alpha)\ell_i^\alpha = A_i(1-\alpha)\left[A_i\alpha\left(\frac{\psi}{w_s}\right)^{\psi}\left(\frac{1-\psi}{w_u}\right)^{1-\psi}\right]^{\frac{\alpha}{1-\alpha}}
\end{equation*}

\end{document}
