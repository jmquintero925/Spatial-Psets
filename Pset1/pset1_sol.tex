\documentclass[12pt]{article}
\usepackage{amsmath,amsfonts,bbm}
\usepackage{fancyhdr,enumitem,xcolor,subcaption}
\usepackage{graphicx} % Allows including images
\usepackage[left=3cm, right=3cm, top =2cm, bottom = 3cm]{geometry}
% \newcommand{\}{}


\pagestyle{plain}
%\setcounter{secnumdepth}{0}
\pagestyle{fancy} 
\rhead{Winter 2023} 
\chead{} 
\lhead{Econ 33550 - Spatial Economics} 
\lfoot{} 
\cfoot{\thepage} 
\rfoot{} 


\title{Spatial Economics - Problem Set I}
\author{Jose M. Quintero \and Jun Wong \and Rachel Williams}

\begin{document}

\maketitle

\section{Introduction}

Some blah blah of what are we going to do, summarize the ingredients of the model, mention the data, limitations of the data if any, our counterfactual and main findings. This should be relatively fast. 

\section{Analysis of Chicago}

A lot of blah, but most critical excellent maps should go here! Beautiful and colorful. 

\section{The Model}


\subsection{Households}
A household is characterized by the neighborhood where she resides $i$ and where she works $j$. The household consume final good $c_{ij}$, land $H_{ij}$. Their utility is enhanced by neighborhood-specific amenities $B_i(R_i)$. Agents additionally have an idiosyncratic shock to their preferences $z_{ij}$ and $\kappa_{ij}\geq 1$ is the disutility of commuting from $i$ to $j$.  \\ 
\begin{equation}
    U_{ij}(z_{ij}) = \frac{B_iz_{ij}}{\kappa_{ij}}\left(\frac{c_{ij}}{\beta}\right)^{\beta}\left(\frac{H_{ij}}{1-\beta}\right)^{1-\beta}
\end{equation}
Following the trade literature, we assume that the shock is distributed Fréchet 
\begin{align*}
    \textcolor{blue}{F(s_{ij})} &\textcolor{blue}{= e^{\lambda_{ij}s_{ij}^{-\theta}}} & \textcolor{red}{F(s_{ij})} &\textcolor{red}{= e^{\lambda^o_{i}\lambda^d_js_{ij}^{-\theta}}}
\end{align*}
with $\theta>0$. 
The budget restriction for the agent is 
\begin{equation*}
    c + q_i^{r}H_{ij} = w_j
\end{equation*}
where I am normalizing the price of the consumption good and assume it is freely tradable within the city. Using profit maximization since the utility is a Cobb-Douglas, then the optimal decisions for $c$ and $H$ are  
\begin{align*}
    c_{ij} &= \beta w_j & H_{ij}&= \boxed{ \frac{(1-\beta)w_j}{q_i^{r}}}
\end{align*}
Then the utility for an agent living in neighborhood $i$, working in $j$ with a preference shock is 
\begin{equation}
    U_{ij}(z_{ij}) = \frac{B_iz_{ij}w_j}{\kappa_{ij}q_{i}^{1-\beta}}
\end{equation}
The intuition of the equation is pretty standard. Higher wages yield higher utility, higher commuting cost decreases utility, amenities increase utility, prices decrease utility and the preference shock increases utility. 

\subsection{Commuting Flows}
First, notice that the utility is also distributed Fréchet, 
\begin{align*}
    \Pr\left(U_{ij}\leq u\right) &= \Pr\left(\frac{B_iz_{ij}w_j}{\kappa_{ij}q_{i}^{1-\beta}}\leq u\right) 
    = \Pr\left( z_{ij}\leq \frac{u\kappa_{ij}q_i^{1-\beta}}{B_iw_j}\right) \\ 
    &= \exp\left(-\lambda_{ij}\left(\frac{B_iz_{ij}w_j}{\kappa_{ij}q_{i}^{1-\beta}}\right)^\theta u^{-\theta}\right) \\ 
    &= \exp\left(-\Phi_{ij}u^{-\theta}\right) = G_{ij}(u)
\end{align*}
Given this fact, the probability that an agent chooses to live in location $i$ and work in location $j$ is 
\begin{align*}
    \Pr\left(U_{ij}\geq \max_{r,s} U_{rs}\right) &= \Pr\left(U_{ij}\geq \max_{s} U_{is}\right)\Pr\left(U_{ij}\geq \max_{r} U_{kr},\quad\forall k\right) \\ 
    &= \prod_{s\neq i}\Pr\left( U_{ij}\geq U_{is}\right)\left[\prod_{r\neq j}\prod_{k}\Pr\left(U_{ij}\geq U_{kr}\right)\right] \\
    &= \int_0^{\infty}\theta\Phi_{ij}u^{-(\theta+1)}\prod_{s}\prod_{r}e^{-\Phi_{rs}u^{-\theta}} \mathrm{d}u \\ 
    &= \int_0^{\infty}\theta\Phi_{ij}u^{-(\theta+1)}e^{-\sum_{r}\sum_{s}\Phi_{rs}u^{-\theta}} \mathrm{d}u \\ 
    &= \frac{\Phi_{ij}}{\Phi}\int_0^{\infty}\theta\Phi u^{-(\theta+1)}e^{-\Phi u^{-\theta}} \mathrm{d}u \\ 
    &= \frac{\Phi_{ij}}{\Phi} = \pi_{ij}
\end{align*}
where $\Phi=\sum_{r}\sum_{s}\Phi_{rs}$

\subsection{Firms}

Firms produce a tradable good using a combination of labor $L_j$ and land $H_j^p$. To keep production simple, assume that the firms combine both inputs using a Cobb-Douglas production function with constant returns to scale
\begin{equation}
    Y_j = A_jL_j^\alpha \left(H_j^p\right)^{1-\alpha}, \quad\alpha\in(0,1)
\end{equation}
Let $y_t$ be the normalized output per unit of land, and $w_j$ the wage paid in location $j$. Then, the optimization problem for the firm is 
\begin{equation*}
    \max_{\ell_j} A_j\ell_j^\alpha - w_j\ell_j
\end{equation*}
The FOC implies the following wage
\begin{equation}
    w_j = \alpha A_j\ell_j^{\alpha-1}
\end{equation}
Assuming that $1-\alpha>\gamma$\footnote{$1-\alpha$ is the price elasticity of land. This assumption ensures that the agglomeration spillovers governed by $\gamma$ do not overtake the congestion effects.}, the labor demand function for location $j$ is decreasing and has the form: 
\begin{equation*}
    L_j = \left(\frac{A_j\alpha}{w_j}\right)^{\frac{1}{1-\alpha}}H_j^p
\end{equation*}
Finally, as the firm has constant returns to scale, it will rent land until it hits zero profits. Then the demand for land it 
\begin{align*}
    q_j &= \frac{Y_j - w_jL_j}{H^p_j} \\ 
    &= (1-\alpha)A_j\ell_j^{\alpha} \\ 
    &= \boxed{(1-\alpha)A_j\left(\frac{\alpha A_j}{w_j}\right)^{\frac{\alpha}{1-\alpha}}}
\end{align*}

\subsection{Market Clearing Conditions}
For this section, we borrow the structure from the Berlin paper. Let $\theta_j$ be the share of land use for production. Then, for there not to be arbitrage, it has to be true that 
\begin{equation*}
    \theta_i = \begin{cases}
    1 &\mbox{if } q_i^p>q_i^r \\ 
    \in[0,1] &\mbox{if } q_i^p=q_i^r \\
    0 &\mbox{if } q_i^p<q_i^r 
    \end{cases}
\end{equation*}

\subsection{Extension: Two types}
Now consider there are two types of agents that are imperfect substitutes to the firm. Specifically, the firm combines them with a Cobb-Douglas technology with 

\end{document}
