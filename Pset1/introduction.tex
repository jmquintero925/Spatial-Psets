% INTRODUCTION 

Chicago is a widely studied city---it is unique in its spatial distribution of people and economic activity. Indeed, seminal work on city structure and segregation started from the examination of Chicago (CITE). The stark differences in economic activity and well-being across space in the city documented since XXX has persisted over time. Figure X shows the distribution of income across space in XXX and in 2010. Here, we build a quantitative spatial model to explain what fundamental forces drive these differences across space in Chicago in the cross-section.\footnote{Other than the fact that this assignment is a static exercise, we argue that dynamics do not add much to understanding the persistent spatial differences. If dynamics mattered more, we might expect the spatial distribution of activity to change.} We extend this model to include the agglomeration of two types of agents to study how including heterogeneous agents affects XXX. We are particularly interested in how different types of people sort across space in face of an amenity shock. We analyze the effects of the Obama Presidential Library construction in East Hyde Park, a relatively wealthier neighborhood in the south side of Chicago. We show that the implications of this is (DIFFERENT OR SAME) across the two models.

%% JW: So I think what would be really interesting is the discussion of what happens if we shocked a place with the lowest income with amenities vs shocking EHP?

% i usually hate these paragraphs. let me know how you feel.
In this introduction, we describe some diagnostics about the city of Chicago to motivate the model that we build in Section XX. Section YY describes the data and Section ZZ shows the results. Section BLAH concludes.

\paragraph{Diagnostics}

We live in a segmented city: the north and northeast of the city are wealthier and has a disproportionate share of the White population; while the west and southwest of the city are poorer and has a disproportionate share of the Black population (Figure XX and XX). Housing prices similarly reflect this concentration of income in the northeast of Chicago (Figure XX). 

Employment is overwhelmingly concentrated in the downtown area, which offers the highest wages (Figure XX). Unsurprisingly, commuting probability from each neighborhood to downtown Chicago is correlated with average income (Figure XX). This is despite some neighborhoods having lower transport costs to the center of the city compared to others (Figure XX). 

Figure XX presents the commuting patterns of one of the poorest neighborhoods in the city, Englewood. While within Englewood, most of the commuters are going to downtown Chicago, we find that 24\% of these commuters work low-wage jobs. This stands in stark contrast to the commuting pattern of Lincoln Park---where the gravity remains in downtown Chicago. However, only 6\% of these commutes consist low-wage earners. Figure XX shows the share of commuters that have low-wage jobs by origin neighborhood. Unsurprisingly, low-wage earners reside primarily in the west and southwest of Chicago. 

Finally, we focus on the commuting characterisics of East Hyde Park. East Hyde Park is unique in that it is a high-income neighborhood located next to lower-income neighborhoods such as Woodlawn.\footnote{East Hyde Park is in the 95th percentile of the income distribution in Chicago.} Residents in East Hyde Park commute mostly to Hyde Park and downtown Chicago. It is not a neighborhood with a lot of employment opportunities: Consider its neighbor, Hyde Park, which attracts up to over 30\% of the commuters from certain neighborhoods. In contrast, East Hyde Park attracts at most 1\% the commuters from any given neighborhood. 